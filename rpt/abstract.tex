\documentclass{article}
\begin{document}
\textbf{Abstract}
The rapid accumulation of conprehensive whole genome sequence(WGS) and brain structure image data mandates the development of analysical procedures capable of utilizing both genetic and image information for the dection of predicative biomarkers for CNS diseases of interest such as Alzheimer. Both types of profiles however, poses "curse of dimensionality" due to methodology design. challage We describe our construction of probabilistic atlases that store detailed information on how the brain varies across age and gender, across time, in health and disease, and in large human populations. Specifically, we introduce a mathematical framework based on covariant partial differential equations (PDEs), pull-backs of mappings under harmonic flows, and high-dimensional random tensor fields to encode variations in cortical patterning, asymmetry and tissue distribution in a population-based brain image database (N =94 scans).
\end{document}