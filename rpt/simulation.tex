\documentclass[twocolumn]{article}

\usepackage{syntonly}
%\syntaxonly
\usepackage{amsmath}

\pagestyle{headings}

\author{Xiaoran Tong}

\begin{document}
\title{A non-parametric method for joint association analysis of sequencing and imaging data}
\maketitle

\section{Simulation}

Each iteration run choose a pair of testing units for genomc and image profiles. The testing unit for WGS profile is a randomly chosen gene from the GRC38 list of genomic features; for neuroimage profile, the testing units are regions of 512 vertices randomly located in the white matter (WM) surface reconstructed from MRI. The genomic and vertex effects $\beta_G$ are then generated by randomly picking out 5\% of the variants (e.g. SNPs in a gene, vertices in a WM region) in both testing units and assigning effects from $N(0,1)$. Given the genomic profile $X_G$ and image profile $X_V$ of all subjects, two continuous response were calculated additively as the product of the profiles and their correspoinding effects, which can be written as:
\[Y_G = X_G \beta_G \]
\[Y_V = X_V \beta_V \]


%\begin{bibliography}
%\end{bibliography}

\end{document}
