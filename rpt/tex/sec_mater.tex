\section{Material}
The next generation sequencing (NGS) and magnetic resonance imaging (MRI) data were obtained from Alzheimer’s Disease Neuroimaging Initiative (ADNI). A totol of 808 subjects at the screening and baseline of ADNI1 and ADNI2 study have both profiles available, alongside with disease diagnosis, demographics, and the genotype of APOE $\epsilon$4. 

The structure MRI data first went throught a series of processing including special registration, skull stripping, cortical/subcortical segmentation, white/gray matter segregation, vexel intensity normalization, reconstruction of cortex, cortex registration, and cortex paceration. The entire pipline is implemented by \FS - a neuroimage analysis package developed by Fisher and Dale et.al. \cite{FS:Intro}, and currently maintained by \textit{the Laboratory for Computational Neuroimaging (LCN)}  at \textit {the Athinoula A. Martinos Center for Biomedical Imaging}. \FS is freely distributed online (\url{http://surfer.nmr.mgh.harvard.edu}). The reconstructed cortex is spaned by $327680$ vertices in 3D space. Each vertex is treated as a image variant, with a number of geometrical attribute attatched to it, such as the coordinate of the vertex, the gray matter thickness, average curvature, local area and volume around its vicinity. Currently we took the gray matter thickness as the value of each vertex. The last processing step of \FS -- cortex paceration, divides the cortical surface into 68 anatomical regions. For real data analysis, these regions are taken as testing units, for simulation study, small ovals of $512$ vertices (mean diametter=28mm) were randomly picked from the cortex as testing units.

The NGS data has gone through rigorous quality control during variant calling process, thus the WGS data from ADNI do not require intensive processing. The testing units for both real data and simulation study are gene based. The chromosome location of known genes were queried from the table of genomic features of human reference genome assemble version 38, maintained by Genome Reference Consortium (GRCh38). An extra 5k flanking region were attached to both ends of a gene when group vairants in a testing unit. Despit the added flanking region, some unit contains no genomic variant, and they were excluded from further opertations. As an result, there are $40,039$ primary and alternative gene assembles eligible for the subsequent study.