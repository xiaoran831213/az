\section{Material}
Whole genome sequencing (WGS) and magnetic resonance imaging (MRI) data were obtained from Alzheimer’s Disease Neuroimaging Initiative (ADNI). A totol of 808 subjects at the screening and baseline of ADNI1 and ADNI2 study have both types of profiles sampled, alongside with Alzheimer Disease diagnosis, demographics, and the genotype of APOE $\epsilon$4.

The MRI data first went throught a series of preprocessing pipeline composed of special registration, skull stripping, cortical/subcortical segmentation, white/gray matter segregation, vexel intensity normalization, reconstruction of brain surface, surface registration and virtual anatomical paceration. The whole pipeline is supported by \emph{FreeSurfer} - a software initiated by [Dale and Fischl et.al.] and currently maintained by \textit{the Laboratory for Computational Neuroimaging (LCN)}  at \textit {the Athinoula A. Martinos Center for Biomedical Imaging}. The preprocessing produces a 3D brain surface, with each hamersphere expressed by $10\times2^16$ vertices connected by triangles, and vitually segemented into 34 anatomical regions [pic ?]. For each vertex, besides its location in the 3D space, statistics ascribing the tissure at its close vicility were also calculated, among which the most important properties are gray matter thickness and sulcity (positive means the vertex is in a gyrus, otherwise in a sulcus). For real data analysis, the 68 anatomical regions are treated as testing unit; for a single iteration in a simulation study, small ovals of $512=2^9$ vertices (mean diametter=28mm) were randomly picked from the whole surface.

Rigorous quality control had been done by ADNI during variant calling process, thus the WGS data from ADNI do not require intensive preprocessing. For genomic profile, the testing unit for both real data and simulation study are based on gene. The chromosome location of known genes were queried from the table of genomic features of human reference genome assemble version 38, maintained by Genome Reference Consortium (GRCh38). An extra 5k flanking basepairs were attached to both ends of a gene when searching for vairants in a testing unit. Despit the added flanking region, some unit contains no genomic variant, and they were excluded from further opertations.