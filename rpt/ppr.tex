\documentclass[12pt]{article}

\usepackage{geometry}
\geometry{a4paper, left=20mm, right=20mm, top=20mm, bottom=20mm}
\setlength{\parindent}{0pt}

\usepackage{hyperref}

\usepackage{amsmath}
\allowdisplaybreaks
\newcommand{\numeq}{\addtocounter{equation}{1}\tag{\theequation}}

\usepackage{amssymb}

\usepackage[utf8]{inputenc}
\usepackage[backend=biber,style=numeric,sorting=none,]{biblatex}
\addbibresource{ref.bib}

\usepackage{graphicx}

% aliasis
% FreeSurfer from Havord Unv.
\newcommand{\FS}{\href{http://surfer.nmr.mgh.harvard.edu}{\textbf{FreeSurfer}}} 

% encoders
% vector or matrix
\newcommand{\vecEC}[1]{\boldsymbol{#1}}

% decoders
\newcommand{\vecDC}[1]{\boldsymbol{\tilde{#1}}}

\newcommand{\xVO}{\boldsymbol{x}}         % the x vector, original
\newcommand{\xVR}{\boldsymbol{\tilde{x}}} % the x vector, recovered
\newcommand{\xSO}{x}                      % the x scaler, original
\newcommand{\xSR}{\tilde{x}}              % the x scaler, recovered

% the eta vector
\newcommand{\etaEC}{\vecEC{\eta}}                % generic encoder
\newcommand{\etaEi}{\WEC_i^{d_{i+1} \times d_i}} % encoder layer i
\newcommand{\etaDC}{\vecDC{\eta}}                % generic decoder
\newcommand{\etaDi}{\WDC_i^{d_i \times d_{i+1}}} % decoder layer i

% the W matrix
\newcommand{\WEC}{\vecEC{W}}                   % generic encoder
\newcommand{\WEi}{\WEC_i^{d_{i+1} \times d_i}} % encoder layer i
\newcommand{\WEI}[3]{\WEC_{#1}^{d_{#2} \times d_{#3}}} % decoder layer #i
\newcommand{\WEIt}[3]{\WEC_{#1}^{d_{#2} \times d_{#3}\prime}} % decoder layer #i, transposed
\newcommand{\WDC}{\vecDC{W}}                   % generic decoder
\newcommand{\WDi}{\WDC_i^{d_i \times d_{i+1}}} % decoder layer #i
\newcommand{\WDI}[3]{\WDC_{#1}^{d_{#2} \times d_{#3}}} % decoder layer #i
\newcommand{\WDIt}[3]{\WDC_{#1}^{d_{#2} \times d_{#3}\prime}} % decoder layer #i

% the w vector
\newcommand{\wEC}{\vecEC{w}}    % generic encoder
\newcommand{\wEI}[2]{{\wEC_{#1}^{1 \times d_{#2}}}}
\newcommand{\wDC}{\vecDC{w}}    % generic decoder
\newcommand{\wDI}[2]{{\wDC_{#1}^{1 \times d_{#2}}}}
\newcommand{\wDIt}[2]{{\wDC_{#1}^{1 \times d_{#2}\prime}}}

% the b vector
\newcommand{\bEC}{\vecEC{b}}    % generic encoder
\newcommand{\bEi}{\bEC_i^{d_i}} % encoder layer i
\newcommand{\bEI}[2]{\bEC_{#1}^{d_{#2}}} % encoder layer i
\newcommand{\bDC}{\vecDC{b}}    % generic decoder
\newcommand{\bDi}{\bDC_i^{d_i}} % encoder layer i
\newcommand{\bDI}[2]{\bDC_{#1}^{d_{#2}}} % encoder layer i

% the x vector
\newcommand{\xEC}{\vecEC{x}}    % generic encoder
\newcommand{\xDC}{\vecDC{x}}    % generic decoder
% the X matrix
\newcommand{\XEC}{\vecEC{X}}    % generic encoder
\newcommand{\XDC}{\vecDC{X}}    % generic decoder

% the y_hat vector
\newcommand{\yHT}{\boldsymbol{\hat{y}}}
\newcommand{\YHT}{\boldsymbol{\hat{Y}}}

% the z vector
\newcommand{\zEC}{\vecEC{z}}    % generic encoder
\newcommand{\zDC}{\vecDC{z}}    % generic decoder

% I/O for decoder layer
\newcommand{\iDi}{\zDC_{i+1}^{d_{i+1}}}
\newcommand{\zEI}[2]{{\zEC_{#1}^{d_{#2}}}}
\newcommand{\zEIt}[2]{{\zEC_{#1}^{d_{#2}\prime}}}
\newcommand{\oDi}{\zDC_i^{d_i}}
\newcommand{\zDI}[2]{{\zDC_{#1}^{d_{#2}}}}
\newcommand{\zDIt}[2]{{\zDC_{#1}^{d_{#2}\prime}}}

% the vector of ones
\newcommand{\one}{\boldsymbol{1}}         % the z vector in encoders
% the diagnal matrix
\newcommand{\I}[1]{\boldsymbol{I}^{#1}}

% parameters in the neural network
\newcommand{\Par}{\boldsymbol{\Theta}} % the parameters
\newcommand{\pEC}{\boldsymbol{\theta}} % the parameters in the stacked autoencoder
\newcommand{\pDC}{\boldsymbol{\tilde{\theta}}} % the parameters in the decoder

% Loss function in Cross Entropy form
\newcommand{\LCE}[2]{#1\log{#2} + (1 - #1)\log{(1 - #2)}}

% derivative
\newcommand{\DRV}[2]{\frac{d #1}{d #2}}        % derivative
\newcommand{\DRC}[3]{\DRV{#1}{#2}\DRV{#2}{#3}} % chained derivative
\newcommand{\PDV}[2]{\frac{\partial #1}{\partial #2}} % paritial derivative
\newcommand{\PDC}[3]{\PDV{#1}{#2}\PDV{#2}{#3}}        % chained

% invers logit, aka. sigmoid function
\newcommand{\SGM}[1]{\frac{1}{1+e^{-#1}}}

% assign to diagnoral
\newcommand{\diag}[1]{\text{diag}(#1)}

\pagestyle{headings}

\author{Xiaoran Tong}

\begin{document}
\title{An Joint Association Analysis Method for Genomic Sequencing and Neuroimaging Data}
\maketitle

\begin{abstract}
The next generation sequencing (NGS) and medical imaging technology gave rise to a growing wealth of genomic and image profiles. While offering fresh opportunities to detect associations between the genome and pcomplex diseases, these dense data also highlight a series of new and existing methodological challenges. As an extra multivariate biomarker, the image profile complicate the statistical modeling because of the unknown interaction between genomic and image variants. The rare genomic variants, and its large number, remain to threat the statistical power. The ``curse of dimensionality'' comes along with the high capacity of the image profile, and more pronounced due to the large number of combinations of two profiles. We propose a method, which builds a series of autoendocers and use them to abstract compact yet informative features from the high dimensional image profile. The method groups and aggregates rare genomic variants for higher statistcial power, and despite an image variant never being ``rare'', the same strategy is applied to image profile as well, which reduces redundant test of highly correlated variants. Finaly, the method uses a joint similarity U statistic to detect associations among genomc, image and phenotype profiles involving unspecified interaction. The simulation study put the robustness and versatility into test, under various scenarios of model misspecification and profile distribution, it also demenstrated the power gained by replacing raw images with the autoencoder abstracted compact features, and by the grouping and aggregation of both types of variants. Lastly, we applied the method to the actual diagnosis from Alzheimer's disease Neuroimaging Initiative (ADNI), strong interaction effect of the two profiles is detected when the genomic profile alone weakly associated with the risk of disease.
\end{abstract}

\section{Introduction}
The decade long search of casual variant by genome wide associatio analysis (GWA) hasn't been satisfying. So far GWA hardly find any single nucleotite variant (SNV) with an large enough effect to act as a stand along necessary cause of any complex diseases. Although a large number of statistically significant common variants were indeed identified by GWA, only a moderate fraction of heritability have been explained by the totality of these finding\cite{GWA1, GWA2}. Despite the setback, human genome is still an intriging source of curiosity owning to its intrisic advantages. When viewed as an exposure, genetic polymophisim is consistant throughout an individual's life course and most types of organism, saving the complication of statistical modeling. Also, as one of the fundermental causes of all biological processes, genomic polymophisim as an exposure, is not suseptable to reverse causality that troubles most non-experimantal designs. From a population perspective, the occurance of genetic variation is random, mimicing a random assignment of exposure in an quasi-experiment, which in turn can be exploited to infer non-genetic effect through an intrumental variable approach, embodied by a Mandilian Randomization design\cite{MR1, MR2}. These features keeps genomic analysis a promising tool for weak effect inference in a complex casual netowrk.

The "common variant, common disease (CVCD)" notion led to the GWA era, as its alternative, the "rare variant, common disease (RVCD)" hypothesis states that the "missing heritability" inexplanable by GWA findings could be attributed to unobserved rare variants of moderate to large effect \cite{RVCD1}. In other words, the so called "whole genome" in GWA was not yet up to its name. The Next Generation Sequencing (NGS) projects, growing in both number and scale over the last decade, facilitated the trial of rare variants. However, as the NGS data keep stockpiling, it also poses a number of new methodological challange. For one, the variants in a NGS profile is much more densely located in the genome, than a typical GWA profile typed by microarrays, which poses serious computational and multiple testing issue should the traditional per-variant screening procedures were applied. Also, as expected, most of the newly called rare variants come with minor allele frequencies (MAF) close to 0. As a direct consequence, such lack of heterogeneity in genotype poses a heavy toll on the statistical power for studies of moderate and smaller sample size. In the epidemiology sence, the number of exposed subjects (either cases or controls) is too small to draw any meaningful inference. So far the most widely accpeted solution is signal aggregation, that is, instead of screening the whole profile one variant after another, one first assign the variants to groups accroding to certain criteria, and subsequently all the variants in one group are test as one unit, by aggregating their signal in a certain way. Commanly, the aggregation can be achieved by collapsing the group into a single variable \cite{Burden1}, or by jointly testing all the variants together \cite{HWU, SKT}. Signal aggregation could be post-screening as well, done by first perform a traditional per-variant GWA screening, then the test statistics (e.g. odds ratio) or its derivation (e.g. the p-values) of a group are combined into a single statistics \cite{Dai:2015, plink1}. Through grouping and aggregation, the number of hypothesis testing reduces to the number of groups, and the heterogeniety of genotype is enchanced to the probability of any variation of the group member. A difficulty comes along with the approach however, is the choice of groupingcriteria. The most popular grouping schemes refer prior knowledge of biological function, resulting in gene or pathway based grouping. Some more debatable schemes rely on the physics of the genome, such as binning the genome by every few kb[?] or by a threshold of linkage disequilibrium (LD) \cite{plink1}. The function based grouping is less subjective and better suited for later interpretation, but for the same reason it is also somewhat a ad-hoc approach, since the ultimate goal of grouping these variants by function, is itself the inference of certain function. Another downside of functional grouping is that it can not exhaustively cover the whole genome, because some variants do not fall into any protein coding gene or known pathway. The physical grouping, on the other hand, could comprehensively cover the entire genome, but is somewhat arbitrary, because the the optimal bin size, LD threshold, and starting position of segmentation, are not pre-known by the investegators. The analytical outcome could also be drastcially different if a different setting was choosen. Aside from signal aggregation, another trend is to form huge, multi-site cohort for large sample such as GIANT and ARIC, or to perform meta-analysis for a large number of published result \cite{META1}, so the sheer number of the observations could overcome the issue of multip testing and low MAF. Now it is not uncommon to apply a brutle per-variant scan for a cohort of more than a thousand subjects. However, the curse of weak effect and low heritability is still lingering, despite frequently reported rare variants reaching statistical significance. The general picture, is still a lack of methodology of moderate prediction power for complexe health outcomes.

The reasons of genetic effects being weak, is in many way like any other fundermantal cause in a casual network, such as social economy status, race/ethnicity and birth place. While being persistant and relatively free from feedback loops, this kind of causes have to reach the final outcomes through a huge "blackbox". As a result, solely altering one dimension of the exposure while leaving the other causes and modifiers untouched won't change the outcome by large. Naturely one would seek amending forces in attempt to unfold the "blackbox", which can be approached from both ends of the casual network. From the downstream end, the bottom-up approach revoles around the central dogma, that is, DNA transcript to mRNA, and mRNA translate to Protein, and thus the profiles of interest are expanded into transcriptome and proteome, respectively. The most prominent practises are expression quantitative trait loci (eQTL)\cite{eQTL1, eQTL2} and gene network analysis. Also from the upstream end, an investegator could collect a variety of strong indicators proximate to the health outcome. Typical examples are conventional biomarkers like inflammatory cytokines, c-reactive protein, serum cholesterol and anti-body titers \cite{cytokine1, CRP1}, and non-conventional biomarkers, such as the neuroimage profiles \cite{VWA1, VWA2, VWA3, VWA4} included by our study. To be noted however, unfolding the "blackbox" is no easy task since any intermidiate factor in the casual network loses the simplicity of a fundermental cause. Namely, an indicators is susceptable to feedback loops and confounders, thus a strong biomarker does not equate to a strong cause, hince the separation of a cause and an indicator. Also, unlike genomic variants, intermidicate indicators are no longer static throughout time and space, for example, the transcriptome of the same subject are quite distinct across different types of cell and life cycle. As an matter of fact, eQTL and gene network analysis are largely seen with, and mostly limited to cancer researches. The downstream strong biomarkers are usually considered as the golden standard of diagnose and confirmation, but are arely suitable for long term risk prediction. A difficulty this work trying to address is the joint analysis of both fundermental casuse and intermidiate factors, either from upstream or downstream. Traditional epidemiology relies on regression to tease apart the direct effect of the fundermental cause and the effect mediated throught the intermidiate factors. This is not suitable when the potential causes in question are groups of tens of hundrends genomic variants, and more ever, when the proximate indicators, in our case, the neuroimage profiles, have even higher dimensionality than an average genomic testing unit.

As mentaioned above, alongsided with the NGS data, the neuroimaging profiles have been incroperated into the statistical models for the added oppertunity to detect associations, and are applied with procedured similar to those designed for genomic analysis. The strength of association between neuroimaging profile and neurological disorder are much higher than from the genome to the same disease \textit{per se} since, the deteriation of cortial surface and subcortical tissue is a proximite event to the final outcome in CNS. Similar to GWA, one could also define a image variant for as the target of statistical inference, which is usually an atomic image unit. For the structure MRI, a variant usually takes the form of a cubic voxel in the 3D volumn spanned by an array of slices [a picture?], which is also an 3D analogy of a square pixel in a 2D plane. The value of voxel taken from the intensity of MRI image, that is, the darkest one has a value of 0, the brightest one is 1. For our study though, the neuroimg profile are not the original MRI and its voxels, but instead we use the 3D cortical surface reconstructed from the structure MRI, performed with the \FS image analysis suite, which is documented and freely available for download online (\url{http://surfer.nmr.mgh.harvard.edu}) \cite{FS:Intro}. Every subject has two cortical surfaces reconstructed, corresponding to the left and right hemispheres, each cortical surface is spanned by 163842 connected vertices, each of which is teated as a variant. The value of a variant(vertex) is no longer the MRI scan brightness, but the gray matter thickness \cite{FS:Tck1, FS:Tck2}, white matter area, gray matter volumn, located around the close proximity of that variant, which is calculated by \FS during the cortical surface reconstruction. The neuroimaging profile, either the brightness of the voxels or the genometry of the vertices, do not suffer the issue of poor signal like the rare genomic variant does, because the values of either type of image units are continuous. Therefore, much like GWA, a per-variant screening can be applied to neuroimage profiles to search for signifiant loci. For analysis using the brightness of voxels from the original MRI slices, this is called voxel-wise analysis (VWA) \cite{VWA1, VWA2, VWA3, VWA4}. As for vertices in the reconstructed surface, \FS comes with its own surface analysis tool which is essentially applying generalized mixed linear regression for each vertex \cite{FS:Anl1, FS:Anl2}, the small area around each vertex is then color coated with the cooresponding $-\log{P}$. A clinical association between neuroanatomical region of interest (ROI) and the health outcome is readly checked by comparing the color pattern of that ROI (Figure \ref{fig:FS1}). In such sense, the \FS surface analysis tool can be called vertex-wise analysis, also abbrevated to VWA. However, the \FS tool is not meat for cohort analysis involving hundreads to thousands participants, because the investigation of color coated surface has to be done by human labor. 
\begin{figure}[h]
\includegraphics[width=\textwidth]{FS_GLM}
\caption[\FS's Vertex-Wise Analysis]{\FS's Vertex-Wise Analysis}
\label{fig:FS1}
\textbf{top:} 40 sample cortical surfaces color coated by gray matter thickness;
\textsf{bottom left:} fitted $thickness - age$ slop at one of the 327684 vertices;
\textsf{bottom right:} cortical surface color coated by $-log{P}$ of the $thickness - age$ slop at each vertex.
\end{figure}
Recognizing that the closely located variants are clearly highly correlated, the grouping and aggregation accepted by genomic analysis can also be used on image profiles to enhance statistical power[Zhu et. al.].
Before the signal aggregation of a testing unit however, a number of procedures could be applied to the unit to enhance statistical power, which usually involves dimension reduction and noise supression. For genomic profiles, a growing trend is to treat it as a discrate sample drawn from a smooth function [Olga V.]. With a well-balanced emphasize on smoothness and goodness of fit, a small expense of profile accuray can reduce the mean differences among the cases as well as among the controls while enlarging the difference between the two group, achieving higher power. Similar priciple can be applied to any type of high dimensional testing unit, as long as we believe the within group difference are most likely contrubited by noises unexplainable with the case/control membership, thus grinding them away with a smoother helps distinguishing the profile of a case from that of a control. A technic commonly adopted by neuroimaging analysis is to burn away trivial details with a gaussian filter while retaining major features by restricting the size of the filter\cite{VWA1, VWA2, VWA3, VWA4}. In this work however, we construct a stacked autoencoder (SA) for each cortical surface testing unit. Unlike the Gaussian filter who uniformly remove a certain level of detail from the image, an SA is actively looking for meaningful high order features and droping trivias.

We are now seeing an increasingly denser genomic profile with large number of rare variants, and the oppertunity to incroporate high dimensional profiles other than genomics. We propose a similarity based U statistics for the association analysis involving more than one high dimensional profiles; we would also try to tackle the dimensionality and low power issue with signal aggregation, and perticularly to the medical image profiles, we will construct an stacked autoencoder to extract major features of lower dimensionality, and lower noise/signal ratio.
\section{Method}

\subsection{Generalized Multivariate Similarity U Statistic}
The goal of our method is to jointly test the possible association among the genomic, cortical surface and the phnotype profiles. The mothod in mind must address several issues. For one, the inclusion of the vertices into the genomic analysis complicates the modeling of association since in most cases an investigator does not know the effect composition in advance, that is, the variation of phenotypes can be attributed to genomic variants or vertices alone, or both, either with or without some unknown type of interaction, mediation or even feedback loops among all three types of profiles. Thus, the method must be sensitive but at the same time robust enough to maintain statistical power when the putative mode is unavoidably missepecified. Also, the value of both genomic and image variants could be generated by distinct and unknown distinct distriubtion types, while the phenotype profile could also be multivariate (e.g. disease diagnosis plus additional demographics and known risk factors), with each element following distinct and uncertain distribution types. Taking these uncertainty into consideration, the test statistic should be versatile enough to counter an admixture of possibally skewed, non-normally distributed data component. More ever, it has to be reasonablly fast in order to deal with the high dimensional profiles comprised of up to tens of thousands genomic polymophisms or vertices (before being replaced with high order features). In this study, we implement the generalized multivariate similarity U statistic (GMSU) postulated by Changshuai et. al. \cite{HWU}. GMSU is computationally efficient, and highly flexible since no distribution assumption will be imposed on genomic and cortical surface profiles, together with a rank based normalization procedure, the U statistic is also invulnerable to multidimensional phenotype with constituents from a mixture of unknown distributions.
To derive the generalized similarity U statistic, three kernel functions are chosen for each of the profiles accordingly. A kernel function measure the similarity between a pair of samples with respect to one of the profiles. Depending on the charisteristics of that profile, the exact form of kernal function can be flexible, as long as it is symmetric and has finite second moment. Thus, measurement $f$ is a valid U kernal function if $f(x_i,x_j)=f(x_j,x_i)$ and $E(f^2(X_1, X_2))<+\infty$ are satisfied. For the current stuty, the functions are chosen according to common practices.

For genomic variants taking values from discrate minor allele count ${0, 1 \textrm{ and } 2}$, the common choice of similarity mesurement is the identical by state (IBS) kernel function
\label{eq:wSG}
\newcommand{\vg}{\pmb{g}}
\newcommand{\GG}{\pmb{G}}
\[ f_G(\vg_{i.}, \vg_{j.}) = \frac{\sum_{m=1}^{|G|}{w_m(2 - |g_{im} - g_{jm}|)}} {2\sum_{m=1}^{|G|}{w_m}}, \]
where $g_{im}$ and $g_{jm}$ is the value of $m$ th. variant in the testing unit (e.g. a gene) taken from the $i$ th. and $j$ th. samples, respectively, and $|G|$ is the dimensionality of the testing unit (e.g. number of polymorphism in a gene). $w_m$ assigns weight to the $m th.$ variant according to \textit{a prior} hypothesis, one example is the minor allele frequency (MAF) based $w_m=\frac{1}{\sqrt{MAF(g_{.m})(1-MAF(g_{.m}))}}$ which gives more emphasize on rare variants. Without any prior knowledge though, the IBS kernal is simplifed to $\frac{\sum_{m = 1}^{|G|}{(2-|g_{im} - g_{jm}|)}}{2|G|}$ by setting $w_m \equiv 1$.

With respect to cortical surface profiles constituted by connected vertices who took continuous values within $[0,1]$, we would used the euclidian distance based function
\label{eq:wSV}
\[  f_V(v_{i.},v_{j.}) = \exp{ [-\frac{\sum_{m=1}^{|V|}{w_m(u_{im}-u_{jm})^2}} {\sum_{m=1}^{|V|}{w_m}}] } \]
to measure the simiarity between sample $i$ and $j$, which is also called a Gaussian kernel function. Here $v_{im}$ and $v_{jm}$ are values of the $m$ th. vertex in the cortical surface testing unit of the $i$ th. and $j$ th. sample, respectively, and $|V|$ denote the number of vertices in the testing unit. The vertices in the cortical surface profile can also be weighted by vector $\boldsymbol{w}.$, but for now we have no prior knowledge of the relative importance of the vertices, the Gaussian kernal function is simplifed to $\exp{[-\frac{\sum_{m=1}^{|V|}{(v_{im}-v_{jm})^2}} {|V|}]}$.

Lastly, for a multivariate phenotype profile whose elements may be drawn from a variety of unknown distributions, we first normalize its elements with rank normal quantile function
\[ q_{im}=\frac{\Phi^{-1}[rank(y_{im})-0.5)]}{|Y|} \]
where $y_{im}$ is the value of the $m$ th. element of the phenotype profile of the $i$ th. sample, and $|Y|$ is the dimensionality of the phenotype (i.e. number of elements). Doing so not only corrects skewed elements, but also bypass the complication of admixed distribution type commanly introduced by a multivariate phenotype. As a result, the phenotype based similarity can be measured in a manner similar to that of cortical surface:
\[ f_Y(q_{i.},q_{j.}) = \exp{[-\frac{\sum_{m=1}^{|Y|}{w_m(q_{im}-q_{jm})^2}} {\sum_{m=1}^{|Y|}{w_m}}]} \]
where $q_{im}$ is the values of the $m$ th. element of the normalized phenotype profile, again with weight $w_m$ denotes the ralative importance of that elements. In case of a phenotype with only one dimension, that is, $|Y|=1$, the similarity measurement simplifys to $\exp{[-(q_i - q_j)^2]}$.
All three kernel functions must be centralized, which is done by substracting the function value at each pair $(i,j)$ with the two marginal mean of all pairs involving $i$ and $j$, respectively, then adding the overall mean of all pairs to it \cite{HWU}. Taking the kernal function of genomic profile as an example, the centralized measurement is
\[ \tilde{f}_G = f_G(\vg_{i.}, \vg_{j.})-\frac{1}{N} \sum_{k=1}^N{f_G(\vg_{i.}, \vg_{k.})}-\frac{1}{N}\sum_{l=1}^N{f_G(\vg_{l.}, \vg_{j.})}+\frac{2}{N(N-1)}\sum_{1 \le k < l \le N}{f_G(\vg_{l.}, \vg_{k.})}\]
where $N$ is number of samples. Finally, the generalized multivariate similarity U statistics is the mean product of all similarity measurement excluding the self-pairs, which is
\[ U_J = \frac{2}{N(N-1)}\sum_{1 \leq i < j \leq N} \tilde{f}_G() \tilde{f}_V() \tilde{f}_Y(). \]
The actual implementation calculated three symmetric $N \times N$ similarity matrices to facilitate later manipulation. The U statistic follows a mixture of $\chi_1^2$ distrubtion, a $p$ value can then be calculated using Davis method \cite{HWU}.

For now, the genomic and vertex similarity are treated U-weight terms, while the phenotype similarity acts as the U-kernel term. In general, the assignment of U-kernal and U-weight terms does not alter the limiting distribution of the final U statistics. More specific association test can be perform by dropping one of the U-weight terms. To see if only the phenotype and genomic profiles are associated, construct the U statistics without vertex similarity term:
\begin{displaymath}
  U^{GY}=\frac{1}{N(N-1)}\sum_{1 \leq i < j \leq N} \tilde{S}_{ij}^{G} \tilde{S}_{ij}^{Y}.
\end{displaymath}
Likewise, dropping the genomic similarity to test if only the phenotype and genomic profiles are uncorrelated:
\begin{displaymath}
  U^{VY}=\frac{1}{N(N-1)}\sum_{1 \leq i < j \leq N} \tilde{S}_{ij}^{V} \tilde{S}_{ij}^{Y}.
\end{displaymath}

\subsection{Stacked Autoencoder}
The stacked autoencoder is an artificial neural network mimicking sentimental visual processing, its purpose is to abstract high order features from the raw image profile. The high order feature not only has lower dimensionality, but is also more relevent  to decision making. Taking our data as an example, being able to see the approximate location and size of the laceration sites in the cortical surface, is far more important than knowning the exact thickness, curvature and coordinates of every vertex in the original profile. Thus, deside the dimension reduction, we also anticipate a power boost for the joint similarity U statistics when the orignal cortical surface profile is replaced with the abstracted features.

An SA is formed by layers of autoencoders stacking on top of each other, hence the name "Stacked". An autoencoder layer performs a linear recombination of the input elements, followed by an element-wise non-linear transformation. Usually, we make sure the output has a lower dimensionality then the input to ensure feature abstraction and dimension reduction. The autoencoder at the $i$ th. layer of the stack is written as:
\begin{equation} \label{eq:AE}
    \zEC_i^{d_i} = s(\WEI{i}{i}{i-1} \zEI{i-1}{i-1} + \bEI{i}{i}) \\
\end{equation},
where $\zEI{i}{i}$ is the layer output and $\zEI{i-1}{i-1}$ is the layer input, which is also the output of the autoencoder from down below, that is, the $i-1$ th. layer in the stack. The cross product between the input vector $\zEI{i}{i}$ and the weight matrix $\WEI{i}{i}{i-1}$ followed by the addition of the offset $\bEI{i}{i}$ achieves the linear recombination of input elements. The superscript $d_{i-1}$ and $d_i$ denote the dimensionality of data and structure parameters of the autoencoder layer. As mentioned before, to ensure feature abstraction and dimension reduction actually happens, $d_i$ is made smaller than $d_{i-1}$. For our method, a autoencoder layer always halve the dimension of its input, that is, $d_i$ = $d_{i-1}/2$. Lastly, the inverse logit is chosen for the elementwise non-linear transformation, thus
\begin{equation} \label{eq:InvLgt}
    s(\etaEC_i^{d_i})     = [\SGM{\eta_{i1}}, \SGM{\eta_{i2}}, \dots, \SGM{\eta_{d_{id_i}}}]^{\prime},
\end{equation}
where $\etaEC_i^{d_i} = \WEI{i}{i}{i-1} \zEI{i-1}{i-1} + \bEI{i}{i}$ is the linear recombination of the input $\zEI{i-1}{i-1}$; the super script $d_i$ denotes its dimensionality and $k = 1, \dots, d_i$ indexes its $d_i$ elements. The "S" shaped inverse logit curve $\SGM{\eta_{ik}}$ resembles the biological activation of the $k$ th. neuron in the $i$ th. layer of visual cortex when the weighted sum of simulations from all $d_{i-1}$ neurons in the previous layer, $\zEI{i-1}{i-1}$, exceeds a threshold. The weight is taken from the $k$ th. row vector of $\WEI{i}{i-1}{i}$,  and the threshold is the negation of $k$ th. elements in the offest vector $\bEI{i}{i}$.

An SA of $M$ layers, of $P$ dimensional raw input $\xEC^P$ and $Q$ dimensional output $\yHT^Q$, is assembled by recursively taking the output of the lower autoencoder layer as the input of the layer above, and ensuring the dimensionality of the output at the top is $Q$.
\begin{equation} \label{eq:ES}
  \begin{split}
    \yHT^Q &= \zEI{M}{M} \\
    \zEI{M  }{M  } &= s(\WEI{M  }{M  }{M-1} \zEI{M-1}{M-1} + \bEI{M  }{M  }) \\
    \zEI{M-1}{M-1} &= s(\WEI{M-1}{M-1}{M-2} \zEI{M-2}{M-2} + \bEI{M-1}{M-1}) \\
    & \quad \quad \quad \quad \vdots \\
    \zEI{i  }{i  } &= s(\WEI{i  }{i  }{i-1} \zEI{i-1}{i-1} + \bEI{i  }{i  }) \\
    & \quad \quad \quad \quad \vdots \\
    \zEI{2  }{2  } &= s(\WEI{2  }{2  }{1  } \zEI{1  }{1  } + \bEI{2  }{2  }) \\
    \zEI{1  }{1  } &= s(\WEI{1  }{1  }{0  } \zEI{0  }{0  } + \bEI{1  }{1  }) \\
    \zEI{0  }{0  } &= \xEC^P,
  \end{split}
\end{equation}
where $\xEC^P$ is the $P$ dimensional raw input of one individual, which is viewed as the output of non-existing $0$ th. autoencoder, with $d_0=P$. Reading from bottom to top, the SA gradually abstracts higher order features from the $P$ dimensional raw input $\xEC^P$, until the dimensionality of the output is as low as $d_M=Q$.

The SA thus constructed is worthless without calibration. One must find the set of structure parameters $\pEC=\{\WEC_1, \bEC_1, \WEC_2, \bEC_2, \dots, \WEC_M, \bEC_M\}$ that best represents the body of knowledge regarding the data, which, in our case, is the knowlege of human cortical surface. Only then the SA is truely capable of abstracting meaningful features out of the raw input instead of haphazardly reducing it into a small but irrelevent output (e.g. a vector of $Q$ random numbers). The ``goodness of abstraction'' can be inferred from the disagreement between the raw input, $\xEC^P$, and its mirrored self, $\xDC^P$, which is the input reconstructed from $yHT^Q$, the abstracted high order features. The rationale is that, the disagreement between $\xEC^P$ and $\xDC^P$ measures how badly the restoration resemble the true original, which indirectly tells us how poorly the encoder had performed, because, a superior abstraction should be less likely to obstruct the recovery effort. Thus, the set of parameter that minimize the difference between the orignal $\xEC^P$ and the reconstructed $xDC^P$ will be considered the optimal configuration of the SA. The calibration guilded by such criteria is called unsupervised training, or unsupervised machine learning. The term ``unsupervised'' states the fact that no external knowledge other than the raw input $xEC^P$ is needed. Instead of tuning the paramters to appeal a certain problem (e.g. logistic regression aims to maximize the classification accuray), unsuersvied learning encourage the SA to manifest itself into an encrypted knowlege of the data of interest, which, in our case, is the knowledge of the human cortial surface. Not requiring labeled data is the greatest strength of unsurpervised learning (e.g. logistic regression requires not just $x$, but pairs of $(x,y)$ to fit $\beta$), which in turn enables a much larger number of sample to contribute to the calibration, and, when new sample become avaible, the training can continue on without reset, mimicing the acceptance of new knowlege. In particular to our method, unsupervised learning make sure all 806 samples could contribute their cortical surface profiles to construct the SA, even if 427 of them cannot enter the joint U statistical analysis because their uncertain disease diagnosis at the baseline.

The new issue on the table is the reconstruction of input, that is, a decoder counterpart of the SA is needed. The most nature way to build the decoder do so is to mirror the encoder structure, so the decoder will also be a stack of unit layers, the number of layers is the same with the SA, each layer also performs linear recombination of its input, followed by a non-linear, element-wise transformation, but, the dimensionality change is in exactly reverse order of the SA. By mirrowing the $i$ th. encoder in the SA, the $i$ th. layer in the decoder stack is
\begin{equation*}
  \zDI{i-1}{i-1} = s(\WDI{i  }{i-1}{i  } \zDI{i  }{i  } + \bDI{i  }{i-1}).
\end{equation*}
 With the layer definition being done, the decoder stack can be assembled in the same way of the SA. Continue with the $M$ layered encoder in \ref{eq:ES}, its decoder counterpart is
\begin{equation} \label{eq:DS}
\begin{split}
  \xDC^P &= \zDI{0}{0} \\
  \zDI{0  }{0  } &= s(\WDI{1  }{0  }{1  } \zDI{1  }{1  } + \bDI{1  }{0  }) \\
  \zDI{1  }{1  } &= s(\WDI{2  }{1  }{2  } \zDI{2  }{2  } + \bDI{2  }{1  }) \\
  & \quad \quad \quad \quad \vdots \\
  \zDI{i-1}{i-1} &= s(\WDI{i  }{i-1}{i  } \zDI{i  }{i  } + \bDI{i  }{i-1}) \\
  & \quad \quad \quad \quad \vdots \\
  \zDI{M-2}{M-2} &= s(\WDI{M-1}{M-2}{M-1} \zDI{M-1}{M-1} + \bDI{M-1}{M-2}) \\
  \zDI{M-1}{M-1} &= s(\WDI{M  }{M-1}{M  } \zDI{M  }{M  } + \bDI{M  }{M-1}) \\
  \zDI{M  }{M  } &= \yHT^Q .
\end{split}
\end{equation}
Reading from bottom to top, the decoder gradually adds details back to the abstracted feature $\yHT^Q$, and eventually produce a retored state of the raw input on its top, denoted by $\xDC^P$. The restoration process is reflected, and driven by the dimenality change from $d_M = Q$ to $d_0 = P$, which is in exact reversed order of the SA. Now with both encoder and decoder stacks ready, the complete cycle of encoding and reconstruction is done by treating the top output of the SA (\ref{eq:ES}), that is, the abstracted code $yHT^P$, as the lowest input of the decoder stack. The combined the structure is
\begin{equation} \label{eq:ED}
\begin{split}
  \xDC^P &= \zDI{0}{0} \\
  \zDI{0  }{0  } &= s(\WDI{1  }{0  }{1  } \zDI{1  }{1  } + \bDI{1}{0  }) \\
  & \quad \quad \quad \quad \vdots \\
  \zDI{M-1}{M-1} &= s(\WDI{M  }{M-1}{M  } \zDI{M  }{M  } + \bDI{M}{M-1}) \\
  \zDI{M  }{M  } &= \yHT = \zEI{M}{M} \\
  \zEI{M  }{M  } &= s(\WEI{M  }{M  }{M-1} \zEI{M-1}{M-1} + \bEI{M}{M  }) \\
  & \quad \quad \quad \quad \vdots \\
  \zEI{1  }{1  } &= s(\WEI{1  }{1  }{0  } \zEI{0  }{0  } + \bEI{1}{1  }) \\
  \zEI{0  }{0  } &= \xEC^P
\end{split}.
\end{equation}
In addition to structure mirroring, a common strategy to train an stacked  autoencoder is to constrain the weight matrix in a decoder layer to be the transpose of its encoder counterpart, that is, by forcing $\WDI{i}{i-1}{i} \equiv \WEIt{i}{i}{i-1}$, the $i$ th. decoder layer become
\begin{equation} \label{eq:CW}
  \zDI{i-1}{i-1} = s(\WEIt{i}{i}{i-1} \zDI{i  }{i  } + \bDI{i  }{i-1}).
\end{equation}
Our method adopts this behavior, because doing so almost halve the number of parameter to be tuned, which is a great boost to the computation, besides, the encoder -- decoder tuples follow the common sense that they are essentially symmetric operations. Most importantly, the constraint encourages the calibration of an optimal SA, instead of a inferior SA coupled with a superior decoder partner. Afterall, our best interest is the abstracted high order features, not the reconstructed input.

The next thing to do is to measure the total disagreement between original profiles and reconstructed profiles for all the samples, which is called the reconstruction loss $L$. For now, the most popular form is cross entrophy
\newcommand{\XECD}{\XEC^{N \times P}}   % encoder input, with dimensions
\newcommand{\XDCD}{\XDC^{N \times P}}   % decoder output, with dimensions
\newcommand{\YHTD}{\YHT^{N \times P}}   % encoder output, with dimensions
\begin{equation} \label{eq:CE}
  L(\XECD, \XDCD) = -\sum_{j=1}^{N}{\sum_{k=1}^P[x_{jk}\log{\tilde{x}_{jk}}+(1-x_{jk})\log(1-\tilde{x}_{jk})]}.
\end{equation}
The two $N \times P$ matrices $\XECD$ and $\XDCD$ store the original and the reconstructed profiles of all $N$ samples, respectively, with each individual sample indexed by $j=(1, \dots, N)$, and the elements in each sample profile indexed by $k = (1, \dots, P)$. One could view the restoration of $\XEC$ from $YHT$ as an array of binary classification problems, with the true probabilities being $\XEC$, the predicted probability being $\XDC$. The reconstruction loss $L$ closely resembles the deviance of a logistic regression analysis which measures of how badly the fitted model reflects the observed reality. With the reconstruction loss $L$ defined, the calibration of SA become a numerical optimization problem
\[ \Par = \min_{\Par} L(\XDCD, \XECD). \]
With the constraint (\ref{eq:CW}) on the weight matrices in the decoder stack, the set of parameters to be tuned is
\[ \Par = \{\WEI{1}{1}{0}, \bEI{1}{1}, \bDI{1}{0}\} \cup \{\WEI{2}{2}{1}, \bEI{2}{2}, \bDI{2}{1}\} \dots \cup \{\WEI{M}{M}{M-1}, \bEI{M}{M}, \bDI{M}{M-1}\}, \]
whose size is $|\Par| = \sum_{i=1}^M{d_i d_{i-1}} + \sum_{j=1}^M{(d_j + d_{j-1})}$, which is $\sum_{i=1}^M{d_i d_{i-1}}$ parameters less then the non-constrained decoder. The optimization procedure is is done by gradient guided iterative algorithm, which is covered in the appendix section.

The optimization is computational intense, because the number of parameters $|\Par|$ is usually large. When the number of layers in the encoder -- decoders stacks is also large, the computation use to be inhibitively hard, because with the same number of allowed parameters, the complexity of the function represented by a network grows exponentially with the number of layers, that is, a deeper SA has more local minimum in the parameter space for the reconstruction loss $L$ to fall into. Yet, complex function also stands for high flexibility, which makes deeper network enormously intrigging, since a exponentially richer funtion space means a much better chance to find a network that could further reduce $L$ and at the same time produce even more compact abstraction. Deep artificial neuro networks have revived its popularity in recent years, thanks to the break through in its training procedure, which is now popularly dubbed ``deep learning''. For our method, we implement the layer-wise greedy pre-training procedure \cite{DL:DBN1, DL:SDA1}. The idea is to first train each encoder layer separately, then fine tune of the entire structure afterwards. To perform the layer-wise pre-training, the output of $i$ th. autoencoder $\zEC_i$ is not sent to the $i+1$ th. autoencoder like (\ref{eq:ED}) does, but is instead redirected to its decoder counterpart immediately to produce the intermidiate reconstruction $zDC_i$, the encoder--decoder tuple can then be easily trained by minimizing the local reconstruction loss $L_i=L(\zEC_{i-1}, \zDC{i-1})$. The training is easy, becase of the small number of parameters $\Par_i=\{\WEI{i}{i}{i-1}, \bEI{i}{i}, \bDI{i}{i-1}\}$. A total of $M$ such tuples is formed and trained separately like
\newcommand{\ED}[2]
{
  \arraycolsep=1.2pt
  \begin{array}{rcl}
    \zDC_{#1} &=& s(\WDC_{#2}\zDC_{#2} + \bDC_{#2}) \\
    \zDC_{#2} &=& \yHT_{#2} = \zEC_{#2} \\
    \zEC_{#2} &=& s(\WEC_{#2}\zEC_{#1} + \bEC_{#2}) \\
    \Par_{#2} &=& \min_{\Par_#2}L(\zEC_{#1},\zDC_{#1})
  \end{array}
}
\begin{equation}\label{eq:Greedy}
\begin{array}{cc}
  \ED{0}{1} \quad \ED{1}{2} \dots \ED{M-1}{M}
\end{array}
\end{equation}
What the greedy layer-wise pre-training has done is non-randomly initialize the entire structure to a state closer to optimum. After pre-training, all the encoders and the decoders are wired together like \ref{eq:ED} and fine tuned together. The comprehensive fine tuning will reach convergence much faster and less likely to fall into a poor local minimum then a direct training scheme without the pre-training. 

\subsection{Implementation}
In this suty, we will first formulate three similarity kernel function based on the three types of profiles available, which is the genotype in NGS format, the reconstructed cortical surface profile in vertex format, and the phenotype composed of diagnostic status and various demographics. The joint similarity U statistics will then be calculated and tested against the null hypotheis that no association exists between any two of the profiles.
Then, we will use all 806 cortical surface profiles to construct a 4 layered the stacked autoencoder (SA), The SA thus created will be used to abstrat the cortical surface profiles of 327 subject with definite diagnossis (either healthy control or Alzeimer's disease) into the condensed code which is 16 times smaller then the original vertex data. We will then use this condensed code to replace the original cortical surface profile and redo the joint similarity U statistic test.
We will use simulation study to compare the the power performance between the test using original vertex versus the one using encoded vertex. A boost in both computation speed and statistical power is expected, since the encoded cortical surface profile is smaller, and the abstracted high order feature is supposedly more informative than the original vertices which is tangled with trivial noises.

\section{Simulation}
The simulations are based on the real NGS and MRI data of 806 ADNI participants with both profiles. Each iteration run choose a pair of genomic and cortical testing units from the two profiles. The genomic testing unit is a gene with 5kp upstream and downstream flanking window. As for the cortex, a testing unit is a region of 512 vertices randomly picked from the entire cortical surface, which is is roughly an oval of 2.8mm diameter. The genomic effect and vertex effect are simulated by assigning values drawn from standard normal distribution to a certain percentage of the variants randomly selected from a testing unit (e.g. polymophisms in a gene or vertices in an oval cortical region). The purely genomic and vertex based response are then generated as the product of the testing unit with the simulated effect. An additive and an interactive response are also created by adding up the two basic responses, with and without an additional element-wise product of the two. Lastly, we assign some noise to the generated responses. For now we will focuse on continous respones, the simulation study for dichotomous responses will be covered later.

\noindent\textbf{Robustness of the Joint U} \\
The first set of study aims to test the robustness of the joint U statistics $U_J$ under very likely circumstances of model mis-specification. The power performance of the joint U is compared with the two simpler statistics without either genomic or vertex kernel function. The performance under 8 sample size setting and the 4 scenarios of effect composition is shown in Figure \ref{fig:PWR_CNT_KNL}.
\begin{figure}[!htbp]
\label{fig:PWR_CNT_KNL}
\centering
\includegraphics[width=300px]{PWR_CNT_KNL}
\caption{Robustness of the joint U statistic}
\end{figure}
The top row of Figure \ref{fig:PWR_CNT_KNL} shows that, the two parsimonious statistics $U_G$ and $U_V$ performed the best when the underlying effect is indeed purely genomic and vertex based, respectively, but they lost all the power when the actually consistuents of the effect do not concure with their choice of U-kernel functions. In constrast, the joint statistic $U_J$ performed fairly well under both circumstances, which is close to the power when the parsimonious kernel composition is correctly specified. What if the phenotipic variation is actually contributed by both genomic variants and cortical vertices? The bottom row of Figure \ref{fig:PWR_CNT_KNL} shows that, the joint statistic $U_J$ outperformed both $U_G$ and $U_V$ when the effect is additve, either with (Figure \ref{fig:PWR_CNT_KNL} down left) or without (Figure \ref{fig:PWR_CNT_KNL} down right) an additional interaction term.

\noindent \textbf{Grouping and Aggregation of Cortex Vertices} \\
Vertices in a cortex have no ``low MAF'' issue that rare genomic variants have, but the variant grouping and signal aggregation strategy adopted by deep sequencing study may still benefit analysis involving cortial surface. The second study aims to see if an aggregated cortex unit achieve higher power than the per-vertex based VWA followed by FDR (false discovery rate) correction. Comparison of the two strategy is done under the same 8 sample size and 4 effect composition scenarios except those with $U_G$ statistics, since cortex profile is not involved. The result is shown in Figure \ref{fig:PWR_CNT_VWA}.
\begin{figure}[!htbp]
\label{fig:PWR_CNT_VWA}
\centering
\includegraphics[width=300px]{PWR_CNT_VWA}
\caption{Grouping and Aggregation v.s. Vertex-wise Analysis}
\end{figure}
As we can see, under all simulation settings, the aggregated testing unit (solid lines) overpowers the per-vertex based VWA (dashed lines). This gap is only shortened when the sample size grows large. Another interesting speculation is the type I error rate of VWA is lower then the 0.05 threshold while the aggregated unit test is closely aligned to 0.05 (top left). The multiple testing correction is done by false discover rate (FDR) adjustment, which says that the adjusted p-value will be conservative if the tests were not independent. Therefore, a conservative type I error rate refects the fact that closely located vertices are correlated as they are sampling from tightly connected brain tissue. As a result, grouping and signal aggregation is also recommended for cortex profile.

\noindent \textbf{Abstract High Order Features from Cortex Vertices}
The third set of study tests whether the high order features abstracted from the raw cortex profile provides higher statistical power than the raw profile itself. Again the comparisons is done under all settings except those involving $U_G$ that doesn't rely on vertices in the cortex. The result is shown in Figure \ref{fig:PWR_CNT_SAE}.
\begin{figure}[!htbp]
\label{fig:PWR_CNT_SAE}
\centering
\includegraphics[width=300px]{PWR_CNT_SAE}
\caption{high order features v.s. raw vertices}
\end{figure}
In most scenarios, the abstracted features (Figure \ref{fig:PWR_CNT_SAE}, dashed lines) offers more power then the orignal vertices (Figure \ref{fig:PWR_CNT_SAE}, solid lines), and the edge is growing with the sample size become larger. The only exception happened when the sample size is lower then 600, the effect is purely vertex based, and the partially mis-specified joint U statistic is used for the test (Figure \ref{fig:PWR_CNT_SAE}, top right). Since the rejection of null hypothesis counts as type I error when the kernel functions are complete mis-sepcified, the top left panel in \ref{fig:PWR_CNT_SAE} shows that, the use of abstracted features doesn't deviate the type I error rate from the 0.05 threshold.

\textbf{Comparions using Dichotomous phenotypes}
The dichotomous responses is simulated by pluging the continuous responses into the inverse logit function for a set of probabilities, then draw the binary case/control status from these probabilities. The power performace shares very similar patterns under every scenario, albeit poorer then its continuous counterpart. The results is shown in Figure \ref{fig:PWR_BIN_KNL}, \ref{fig:PWR_BIN_VWA} and \ref{fig:PWR_BIN_SAE}.

In general, the simulation studies have so far demenstrated the robustness and versitility of the proposed method when faced with uncertain effect composition and a variety of phenotype distributions. Also shown is the helpfulness of groupping and aggregation strategy used by many rare genomic variant studies over other types of high dimenstional whose variants are not ``rare'' but potentlly correlated. The power boost offered by the stacked autoencoders is not dramatic, but is increasingly more positive when the sample size grows.

\section{Real Data Analysis}
The baseline data of 327 out of 806 participants who has definite diagnosis status entered the analysis. The genomic testing units are still defined by gene. The image testing units are now the 68 functional anatomy regions in the cortex. The 68 sets of vertices are piped into 68 stacked autoencoders trained with all 806 cortex profiles, and the resulting 68 sets of abstracted features are used to construct joint U statistic $U_J$. Among those 327 choosen subjects, 47 of them are diagnosed with either Alzheimer's disease (AD) or dementia, while the rest 280 subjests are healthy controls (CN). The case/control outcomes were first regressed on 7 known risk factors of AD, namely age, gender, race, ethnicity, years of education, marriage status, ever smoking, and APOE $\epsilon$4 haplotype. The regression residuals were then taken as the actual phenotype. For exploratory purpose, for each tuple of gene and cortex region, we also put the other two simplified statistics $U_G$ and $U_V$ into test. The testing resuls of 3 null hypothesis are show in Figure \ref{fig:RDA_PVL}, sorted from left to right by the p-value the joint U statistic $U_J$.
\begin{figure}[!htbp]
\label{fig:RDA_PVL}
\centering
\includegraphics[width=300px]{RDA_LGP}
\caption{p-values of real data analysis}
\end{figure}
Going through all triplets of $U_J$, $_G$ and $U_V$ statistics, we see most of the time, the joint statistic $U_J$ that involves a pair of gene and cortex region (Figure \ref{fig:RDA_PVL} dots), is less significant than the purely vertex based $U_G$ on the same cortex region (Figure \ref{fig:RDA_PVL} diamonds) but more significant than the purely genomic based statistic $U_G$ involves the same gene (Figure \ref{fig:RDA_PVL} cross). Such pattern reflects the facts that genomic effect is weak disease while the cortex profile is a strong indicator of the disease in the brain. From the whole picture, we see the joint statistcs ``borrowing'' information from the cortex profile, such that the p-values of $U_J$ align closer to the more significant $U_V$, and topple it when both $U_V$ and $U_G$ are significant for a pair of gene and cortex region.
The results showed that the vertex based test are statistically more significance then the genome based ones, which is coherent with the fact that the neuron loss and thinning of gray matter is an proxmate indicator of the progression of Alzheimer's , while genomic profile is a remote predictor of a greater uncertainty. A noteworthy phenominium is how the joint U statistic ($U_J$) could "borrow" power from the two simpler genomic ($U_G$) and vertex ($U_V$) statistics, such that for most combinations of gene and M regions the p-value of the joint test aligns closer to the more significant one of the two simpler U statistics, moreover, when both $U_G$ and $U_V$ are moderately significant, $U_J$ will be more so then either of them. In fact, the top 10 tests in table[?] were all from the joint U statistics $U_G$, which is the combination of the most significant WM region - left superiortemporal and the 10 most significant genes from tests $U_Vnnnn$ and $U_G$, respectively. Decection of left superiortemporal by either vertex U or joint U test is consistant with known fact that thinning of this region is a strong indicator for AD diagnosis, which is backed by ample evidences from imaging and atopsy studies [?]. The top genes so far detected couldn't remain statistically significant after multiple-testing correction and didn't contain or close to the top 20 SNPs [ref: AD SNP database] reported so far by GWAS and meta-analysis, they are likely to be due to chance.

20 most significant tuples, according to the smallest p value in each triplet of $U_J$, $U_G$ and $U_V$. 

% latex table generated in R 3.2.3 by xtable 1.7-4 package
% Thu Feb 11 21:19:43 2016
\begin{table}[ht]
\centering
\begin{tabular}{lllll}
  \hline
  cortex & gene & $P_V$ & $P_G$ & $P_X$ \\ 
  \hline
  SFG & IGLV1-44 & 1.68e-11 & 3.51e-04 & 2.77e-13 \\ 
  SFG & NBEAP2 & 1.68e-11 & 1.19e-04 & 4.74e-13 \\ 
  SFG & RPL21P89 & 1.68e-11 & 6.36e-04 & 5.14e-13 \\ 
  SFG & LOC102724504 & 1.68e-11 & 1.41e-03 & 5.56e-13 \\ 
  SFG & CNTNAP3P8 & 1.68e-11 & 1.08e-03 & 6.17e-13 \\ 
  SFG & CDH4 & 1.68e-11 & 4.64e-03 & 6.96e-13 \\ 
  SFG & HNRNPA1P19 & 1.68e-11 & 8.88e-04 & 7.80e-13 \\ 
  SFG & FAM72C & 1.68e-11 & 9.28e-06 & 7.82e-13 \\ 
  SFG & RP11-638L3.1 & 1.68e-11 & 1.41e-01 & 9.49e-13 \\ 
  SFG & CPXM1 & 1.68e-11 & 8.78e-04 & 1.08e-12 \\ 
  SFG & LOC101929612 & 1.68e-11 & 1.44e-02 & 1.15e-12 \\ 
  SFG & LOC100996517 & 1.68e-11 & 6.77e-04 & 1.20e-12 \\ 
  SFG & IGLV5-45 & 1.68e-11 & 3.44e-04 & 1.23e-12 \\ 
  SFG & MIS18BP1 & 1.68e-11 & 4.95e-03 & 1.35e-12 \\ 
  SFG & CDR2 & 1.68e-11 & 1.82e-04 & 1.39e-12 \\ 
  SFG & RPL41P2 & 1.68e-11 & 6.04e-03 & 1.59e-12 \\ 
  SFG & LOC101927737 & 1.68e-11 & 7.20e-03 & 1.60e-12 \\ 
  SFG & IGLV1-47 & 1.68e-11 & 1.44e-02 & 1.60e-12 \\ 
  SFG & IGLV7-46 & 1.68e-11 & 9.15e-04 & 1.69e-12 \\ 
  SFG & ZDHHC15 & 1.68e-11 & 1.56e-03 & 1.73e-12 \\ 
   \hline
\end{tabular}
\caption{Top 20 combinations} 
\label{tb:tp20}
\end{table}



\section{discussion}
The proposed method can effectively combine information from multiple high dimensional data sources of distinct type to achieve higher statistical power, which in our case are the joint signal of genomic and cortex profiles. The major strength of the metod is its robustness and versatility. The robustness is demenstrated by its ability of retaining power that is close to the optimal model specification, when the true effect constitution was not known $a prior$. The versatility is shown by its acceptance of a wide variety of profiles regardless of their distribution. The proposed method also applies variant grouping and signal aggregation strategy to the cortex profiles, which not only considerably boosted the statistcial power over the per-varaint screening procedure, but also save the computation time. With properly assigned kernel functions, the method can also incorporate additional profiles into the analysis, such as other ``omics'' data closer to the upstream genomics, or inflammatory bio-markers closer to the down-stream health outocme, without worrying about many possible high order interactions among more than two or three types of profile.

The method also build one of its components with the high order features abstracted from the cortex profile of eligible samples ($n=327$) using the stacked autoencoders (SA) tained with the whole dataset ($n=806$). The abstracted features not only has lower dimensionality, but also help the methods to achieve slightly higher statistical power. We feel the potent of the deep artifical neural network is only explored at its surface, and the usefulness of the SA opened up intriguing prospects. First, the encoders can go deeper, by increasing the number of layers. Though a deeper SA is harder to train, it is capable of creating more compact yet more meaningful abstraction from the orignal input, subsequently boost the statistical power even further yet lower the computation load. Second, as mentioned before, the SA was trained in an unsupervised manner, which means a bank of data collected not for any particular study can be used to increamentally refine the existing SA, as long as the data collection follows compatible protocols. In our case, the 489 subjects who couldn't enter the real data analysis due to uncertainly in diagnosis, still contributed their cortex profile to train the SAs. Further, in the near future, the rest of the ADNI participants who were not included in this study due to the lack of next generation sequencing genomic profile, can still help to refine the SAs we already have, because every ADNI participants has structure MRI data. Lastly, it is very tempting to use the same training and abstraction approach on the genomic profile, which is also high dimentional and growing in size, and again, with unsupervised training technique, we can utilize a huge wealth of NGS data from collaborators and public database, such as the freely available 1000 genome project data.

These prospects, are not coming without challenges. First regarding the construction of the joint U statistic, it is easy to bring in more high dimensional components, but when an overall association is detected, it is very hard to tease out exactly how each component contributed to that association, and it is also hard to tell the  (other than a simple product form in the simulation) and effect size of that interaction between the components when more than two kernel functions are involved. In the real data analysis, when the joint U statistic $U_J$ turned out to be statistically significant while the two simplified $U_G$ and $_V$ didn't, the interaction between the genomic and cortex profile is guaranteed, but one can not be sure this interaction is associated with the phenotype profile. When all three statistics are significant, we known from the simulation study the interaction between genomic and cortex profiles may, or may not exists. The proposed method is better suited for fast screening of a large number of combinations of multiple high dimensional profiles, but in the end an explicit modeling is still reqiured for the categorization and quantifying of the associations.

Another challenge involves the way variants are grouped, and the way stacked autoencoders are trained and utilized. An autoencoder require constant input dimension. The cortex profile is stable in the number of vertices ($2 \times 1683400$), so are the 67 anatomy regions, if the current and furture samples are registered to the same atlas. We trained 67 stacked autoencoders for each region, but in reality, they are far too coarse to accurately pinpoint the loci in the cortex. For the genomic data, grouping variants by gene is an accepted compromise between accuracy and statistical power, however, the input dimension is not fixedsince the number and location of variants can differ from study to study, also the dimension of a gene can differ from sample to sample due to the indels and copy number variation. Besides, training over $40$ thousands stacked autoencoders is hardly affordable given the intensity of computation even if the greedy layer-wise pretrain techneque is used. So a grouping and training scheme not colluded with any existing functional information is required. Instead of training an SA for any specific gene or cortical region, an SA will be only be trained for an resonable input dimensionality, which is small enough to both satisfy the desired accuracy and ease the computation, but not too small so no meaningful high order feature is contained. Taking the cortex profile as an example, if the manipulation of vertices in the 3D space is not too complicated, one could first realign the vertices to a sphere with uniform spacing without altering the topology of the cortex, after which only one SA will be trained to encrypt the general knowledge of any cortical region mapped to a sphere surface of, for example, 5 degree in both latitude and longitude. For the genomic profile, only one SA will be trained to encrypt the general knowledge of, for example, a 100kb segment any where in the genome except the proximity of centromeres and telomeres. Yet, another complication will surface due to the inclusion of redundant sequences between the nucleotide polymorphisms, making the abstracted feature likely to have even higher dimension than the polymorphism in any 100kb region, unless a very deep stacked autoencoder is successfully trained. What can be guaranteed however, is the abundance of training materials.

There is also rooms to improve the simulation studies. We known the disease in CNS does not alter the genome, but it does so to the cortex, often making visible changes of features. The current simulation only assigned effects to vertices randomly chosen across a testing unit, instead of changing the thickness for a group of connected vertices to creat a visible shrinkage. As an result, the benefit of feature abstraction may be more pronounced in the real data analysis, without the evidence from the simulation study. The desired simulation however, may required intense human input because the change of features must mimic the real life clinical experiences, plus the difficulty to manipulate the vertices in 3D space. A more approachable simulation can be done by assigning the effect to vertices clusterd in existing visible featuers such as the few dozen named sulci and gyri which is already marked by \FS.

Should these challenges be addressed and improvements be realized, we see a greater used of the rich and ever accumulating data, for more powerful inference of the relationship between complex disease and genome.


%\printbibheading
\printbibliography

\section{Appendix}
\subsection{Gradient Descent}
\newcommand{\oi}[2][]{\boldsymbol{o}_{#2}^{#1}}
\newcommand{\si}{\boldsymbol{s}}
\newcommand{\ei}[2][]{\boldsymbol{\eta}_{#2}^{#1}}
\newcommand{\bi}[2][]{\boldsymbol{\theta}_{#2}^{#1}}
The optimization is done by gradient decent. Starting with a randomly initialized assignment of $\Par^t$ at $t=0$, and update it by substracting a small fraction $\gamma$ of the gradient of reconstruction loss $d(\xDC, \xEC)$ with respect to the current assignment ($\PDV{d}{\Par^t}$). 
\[ \Par^{t+1} = \Par^t - \gamma \PDV{d}{\Par^t} \]
The small fraction $\gamma$ is called learning rate, if $\gamma$ is small enough, the loss $d$ will keep dropping. The updating process repeats itself until $d$ cease to drop and the final assignment $\Par^*$ is considered optimal.

The high dimensional gradient $\PDV{d(\xDC, \xEC)}{\Par}$ is calculated through backward propagation (BP) which heavily relies on the chain rule. The loss $d(\xDC, \xEC)$ is a function of $\xEC$ which in our case is the corss-entropy, and $\xDC$ in turn is a function of $\Par$, thus we have:
\begin{equation*}
  \begin{split}
    \PDV{d(\xDC, \xEC)}{\Par} &= \PDV{d(\xDC, \xEC)}{\xDC} \PDV{\xDC}{\Par}\\
    &= [\xEC \oslash \xDC - (\one - \xEC) \oslash (\one - \xDC)] \PDV{\xDC}{\Par} \\
    \PDV{\xDC_j^P}{\Par} &= \PDV{\zDC_{0j}^{d_0}}{\Par},
  \end{split}
\end{equation*}
where $\oslash$ means entry-wise division. 

Recall the symbolic form of any layer in an SA (see \ref{eq:SE}, \ref{eq:SD} and \ref{eq:SA}), regardless of being an encoder or a decoder, the layer output $\oi{l}$ is always a function its own parameters $\bi{l} = \{\WEC_l, \bEC_l\}$ and the output from the layer below, $\oi{l-1}$, that is,
\[ \oi{l} = \si(\ei{l}), \quad \ei{l} = \oi{l-1} \WEC_l + \bEC_l. \]
where $\ei{l} = \oi{l-1} \WEC_l + \bEC_l$ is the linear recombination and offsetting, while $\oi{l} = \si(\ei{l})$ being the non-linear entry-wise transformation which is the inverse logit in most cases. The gradient of a layer's output w.r.t its own parameters and immediate input, $\PDV{\oi{l}}{\bi{l}}$ and $\PDV{\oi{l}}{\oi{l-1}}$, are relatively easy to implement, 
\begin{equation*}
  \PDV{\oi{l}}{\ei{l}} = 
  \left[\begin{array}{cccc}
          \oi{l1}(1-\oi{l1}) & 0                  & \dots  & 0 \\
          0                  & \oi{l2}(1-\oi{l2}) & \dots  & 0 \\
          \vdots             & \vdots             & \ddots & \vdots \\
          0                  & \dots              & \dots  & \oi{lq}(1-\oi{lq})
    \end{array}\right]
\end{equation*}
and with them avaiable, the gradient of the highest layer's output $\oi{2M}$, that is, the reconstructed input $\xDC$, w.r.t parameters and output from any lower layer are recursively derived from top to bottom:
\newcommand{\SEP}{\qquad}
\newcommand{\PDT}[3]{\PDV{#1}{#2}\PDV{#2}{#3}}
\newcommand{\BPI}[4][\zDC_0]{\PDV{#1}{#2} &= \PDC{#1}{#4}{#2} & \qquad \PDV{#1}{#3} &= \PDC{#1}{#4}{#3}}
\newcommand{\DOT}            {             &  \vdots           &                     &  \vdots}
\begin{align*}
  \BPI{\pDC_2}{\zDC_2}{\zDC_1} \\
  \BPI{\pDC_3}{\zDC_3}{\zDC_2} \\
  \DOT \\
  \BPI{\pDC_M}{\zDC_M}{\zDC_{M-1}} \\
  \BPI{\bi{ M  }}{\oi{ M-1}}{\oi{ M  }} \numeq \label{eq:BP} \\
  \BPI{\bi{ M-1}}{\oi{ M-2}}{\oi{ M-1}} \\
  \DOT \\
  \BPI{\bi{ 2  }}{\oi{ 1  }}{\oi{ 2  }} \\
  \BPI{\bi{ 1  }}{\oi{ 0  }}{\oi{ 1  }} \\
\end{align*}

\begin{equation*}
  \PDV{\oi{l}}{\Par_m} = \PDC{\oi{l}}{\oi{l-1}}{\Par_m}
\end{equation*}

 par simple if only one layer is considered. $\zDI{i-1}{i-1}$, $\pDC_{i}=[\WDI{i}{i-1}{i}, \bDI{i}{i-1}]$ and $\zDI{i}{i}$, respectively (see \ref{eq:DS}). The calculation for an encoder layer will be similar, since the structure of an encoder layer is identical to its decoder counterpart except the dimensionality change (see \ref{eq:ES} and \ref{eq:ED}). The derivatives will be calculated separately for each output element $\tilde{z}_{ik} (k=1,\dots,d_{i-1})$, because the inverse logit transformation is applied element-wise. To ease the thought process, the $i$ th. decoder layer is rewritten in the per-element manner,
\begin{equation*}
  \zDI{i-1}{i-1} = 
  \left[\begin{array}{c}
      \tilde{z}_{i-1,1} \\ 
      \tilde{z}_{i-1,2} \\ \vdots \\
      \tilde{z}_{i-1,k} \\ \vdots \\
      \tilde{z}_{i-1,d_{i-1}}
    \end{array}\right]
  = s(\WDI{i}{i-1}{i} \zDI{i}{i} + \bDI{i}{i-1})
  = \left[ \begin{array}{c}
      logit^{-1}(\wDI{i1}{i} \zDI{i}{i} + \tilde{b}_{i1}) \\
      logit^{-1}(\wDI{i2}{i} \zDI{i}{i} + \tilde{b}_{i2}) \\ \vdots \\
      logit^{-1}(\wDI{ik}{i} \zDI{i}{i} + \tilde{b}_{ik}) \\ \vdots \\
      logit^{-1}(\wDI{id_{i-1}}{i} \zDI{i}{i} + \tilde{b}_{id_{i-1}}) \\
    \end{array} \right],
\end{equation*}
where $\wDI{ik}{i}$ is the $k$ th. row vector of weight matrix $\WDI{i}{i-1}{i}$, and $\tilde{b}_{ik}$ is the $k$ th. element of threshold vector $\bDI{i}{i-1}$, with $k=1 \dots d_{i-1}$. The gradients of the $k$ th. element in the output, $\tilde{z}_{i-1,k}$, with respect to its contributing parameters $\pDC_{ik}=[\wDI{ik}{i}, \tilde{b}_{ik}]$ and the input $\zDI{i}{i}$ are
\begin{equation*}
  \begin{split}
    \PDV{\tilde{z}_{i-1,k}}{[\wDI{ik}{i}, \tilde{b}_{ik}]}
    &= \PDC{\tilde{z}_{ik}}{\wDI{ik}{i} \zDI{i}{i} + \tilde{b}_{i2}}{[\wDI{ik}{i}, \tilde{b}_{ik}]} \\
    &= \tilde{z}_{i-1,k}(1-\tilde{z}_{i-1,k})[\zDIt{i}{i},1] \\
    \text{and,} & \\
    \PDV{\tilde{z}_{i-1,k}}{\zDI{i}{i}}
    &= \PDC{\tilde{z}_{ik}}{\wDI{ik}{i} \zDI{i}{i} + \tilde{b}_{ik}}{\zDI{i}{i}} \\
    &= \tilde{z}_{i-1,k}(1-\tilde{z}_{i-1,k})\wDI{ik}{i},
  \end{split}
\end{equation*}
respectively. Notice that, the derivative of $y=logit^{-1}(x)=\SGM{x}$ has a more compact expression using the dependent variable $y$ instead of $x$, since
\begin{equation*}
  \DRV{\SGM{x}}{x} = \frac{1}{1+e^{-x}}\frac{e^{-x}}{1+e^{-x}} = \SGM{x}(1-\SGM{x}),
\end{equation*}
thus, during the invocation of chain rule, we have
\begin{equation*}
  \DRV{\tilde{z}_{i-1,k}}{\wDI{ik}{i} \zDI{i}{i} + \tilde{b}_{ik}}=\tilde{z}_{i-1,k}(1-\tilde{z}_{i-1,k}).
\end{equation*}
Packing up the gradient for each element in the $i$ th. decoder's output into a vector, we have
\begin{equation*}
  \begin{split}
    \arraycolsep=1.4pt\def\arraystretch{1.5}
    \PDV{\zDI{i-1}{i-1}}{\pDC_{i}}
    &= \left[\begin{array}{c}
        \PDV{\tilde{z}_{i-1,1}}{[\wDI{i1}{i}, \tilde{b}_{i1}]} \\
        \PDV{\tilde{z}_{i-1,2}}{[\wDI{i2}{i}, \tilde{b}_{i2}]} \\ \vdots \\
        \PDV{\tilde{z}_{i-1,k}}{[\wDI{ik}{i}, \tilde{b}_{ik}]} \\ \vdots \\
        \PDV{\tilde{z}_{i-1,d_{i-1}}}{[\wDI{id_{i-1}}{i}, \tilde{b}_{id_{i-1}}]} \\
      \end{array} \right]
    = \left[\begin{array}{c}
        \tilde{z}_{i-1,1}(1-\tilde{z}_{i-1,1}) [\zDIt{i}{i}, 1] \\
        \tilde{z}_{i-1,2}(1-\tilde{z}_{i-1,2}) [\zDIt{i}{i}, 1] \\ \vdots \\
        \tilde{z}_{i-1,k}(1-\tilde{z}_{i-1,k}) [\zDIt{i}{i}, 1] \\ \vdots \\
        \tilde{z}_{i-1,d_{i-1}}(1-\tilde{z}_{i-1,d_{i-1}}) [\zDIt{i}{i}, 1] \\
      \end{array} \right] \\
    &= \diag{\zDI{i-1}{i-1}} \diag{\one - \zDI{i-1}{i-1}} [\zDIt{i}{i},1], \\ \text{in a similarly way,} \\
    \PDV{\zDI{i-1}{i-1}}{\zDI{i}{i}}
    &= \left[\begin{array}{c}
        \PDV{\tilde{z}_{i-1,1}}{\zDI{i}{i}} \\
        \PDV{\tilde{z}_{i-1,2}}{\zDI{i}{i}} \\ \vdots \\
        \PDV{\tilde{z}_{i-1,k}}{\zDI{i}{i}} \\ \vdots \\
        \PDV{\tilde{z}_{i-1,d_{i-1}}}{\zDI{i}{i}} \\
      \end{array} \right]
    = \left[\begin{array}{c}
        \tilde{z}_{i-1,1}(1-\tilde{z}_{i-1,1})\wDI{i1}{i} \\
        \tilde{z}_{i-1,2}(1-\tilde{z}_{i-1,2})\wDI{i2}{i} \\ \vdots \\
        \tilde{z}_{i-1,k}(1-\tilde{z}_{i-1,k})\wDI{ik}{i} \\ \vdots \\
        \tilde{z}_{i-1,d_{i-1}}(1-\tilde{z}_{i-1,d_{i-1}})\wDI{id_{i-1}}{i} \\
      \end{array} \right] \\
    &= \diag{\zDI{i-1}{i-1}} \diag{\one - \zDI{i-1}{i-1}} \WDI{i}{i-1}{i}
  \end{split}
\end{equation*}

\begin{equation}\label{eq:GD}
  \begin{split}\
    \begin{array}{rl}
      \textrm{the i th. decoder:} & \begin{array}{rcl}
        \PDV{\zDI{i-1}{i-1}}{\pDC_i} & = & \diag{\zDI{i-1}{i-1}} \diag{\one{} - \zDI{i-1}{i-1}} [\zDIt{i}{i}, 1] \\
        \PDV{\zDI{i-1}{i-1}}{\zDI{i}{i}} & = & \diag{\zDI{i-1}{i-1}} \diag{\one{} - \zDI{i-1}{i-1}} \WDI{i}{i-1}{i} \\
      \end{array} \\ \\
      \textrm{the i th. encoder:} & \begin{array}{rcl}
        \PDV{\zEI{i}{i}}{\pEC_i} & = & \diag{\zEI{i}{i}}\diag{\one - \zEI{i}{i}} [\zEIt{i-1}{i-1}, 1] \\
        \PDV{\zEI{i}{i}}{\zEI{i-1}{i-1}} & = &  \diag{\zEI{i}{i}} \diag{\one - \zEI{i}{i}} \WEI{i}{i}{i-1}\\
      \end{array}
    \end{array}
  \end{split}
\end{equation}
Here $\diag{v}$ means creating a $0$ matrix and asign the vector $v$ to its diagonal. Since the input of a layer is essencially the output of the layer down below, who also has its own structure paremeters and input from the layer even lower, the gradient of the top output, $\zDC_0$, with respect to any lower layers' parameters, can be calculated by recursively invoking the chain rule, that is,

In reverse to the "bottom to top" encoding and decoding procedure, the gradient is propagated from top to bottom, hince the name "backward propagation". Taking the $M$ layered SA and its decoder counterpart together, the total number of parameters to be calibrated is $|\Par| = |\pEC| + |\pDC| = \sum_{i=1}^{M}{(d_i + 1)d_{i-1}} + \sum_{j=M}^{1}{(d_{j-1} + 1)d_j}$. The optimization can be computationally intense because this number is usually huge. One commonly applied strategy for learning an SA is to constrain the weight matrix in the decoder layers to be the transpose of their counterpart in the SA, that is, forcing $\WDI{i}{i-1}{i} \equiv \WEIt{i}{i}{i-1}$ for $i=1 \ldots M$, and on top of the assignment of gradient layed out in \ref{eq:GD} and \ref{eq:BP}, the gradient of the top output $\zDI{0}{0}$ with respect to the weight matrix of any decoder will be absorded by the one in its encoder counterpart, that is, 
\begin{equation*}
  \begin{split}\
    \PDV{\zDC_0}{\WEC_i}^* &= \PDV{\zDC_0}{\WEC_i} + (\PDV{\zDC_0}{\WEC_i})^\prime.
  \end{split}
\end{equation*}
Doing so introduces slightly more computation for each learning step, but at the same time almost halve the number of tuning parameters to $\sum_{i=1}^{M}{(d_{i-1}) \times (d_i - 1)} + \sum_{i=M}^{1}{d_i}$, greatly speed up the convergence of $L(\XDC, \XEC)$.  Beside, the whole structure fits the common sense that, encoding and decoding are essentially symmetric operations. More importantly, the constraint encourages learning of an optimal SA instead of a sub-optimal SA coupled with a powerful decoder on its top, afterall, our best interest is the high order feature abstracted from the raw input, not its reconstruction.
\subsection{Simulation study of Binary Phenotype}

\begin{figure}[!htbp]
\label{fig:PWR_BIN_KNL}
\centering
\includegraphics[width=300px]{img/PWR_BIN_KNL.png}
\caption{Joint U v.s. Parsimonious U statistics (Binary)}
\end{figure}

\begin{figure}[!htbp]
\label{fig:PWR_BIN_VWA}
\centering
\includegraphics[width=300px]{img/PWR_BIN_VWA.png}
\caption{Vertex Grouping \& Aggregation v.s. Vertex-wise Analysis (Binary)}
\end{figure}

\begin{figure}[!htbp]
\label{fig:PWR_BIN_SAE}
\centering
\includegraphics[width=300px]{img/PWR_BIN_SAE.png}
\caption{High order features v.s. original vertices}
\end{figure}

\end{document}
